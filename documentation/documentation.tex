\documentclass[12pt,letterpaper]{article}
\usepackage[utf8]{inputenc}
\usepackage{amsmath}
\usepackage{amsfonts}
\usepackage{amssymb}
\usepackage[left=1.00in, right=1.00in, top=1.00in, bottom=1.00in]{geometry}
\author{Brett Worley}
\title{Skeletal Age Calculator Project Documentation}
\date{version: \today}

\begin{document}

\maketitle
\tableofcontents

\section{Introduction}

This project involves porting over a previously written program into a portable language. The user's main need for the project is to update and port over a previously written program to a modern system independent language. 

\subsection{Requirements}

The basic requirements of the program are as follows.

\begin{enumerate}

\item The program must provide the same functionality as the previous program.
\item The program must be system independent.
\item The program must be validated against the test cases for the previous program.
\item The program will provide an easy to use UI, it shall follow the format of the forms that the users are accustomed to.
\item The program should be extensible.

\end{enumerate}

\section{Objectives}

\subsection{Introduction}

The first part of the project should be porting over the previous code into a portable language and develop a mock up of the UI. The UI should follow the format of the medical forms that are given alongside of the previous code. The new program should also be able to read in the data-set of measurements, instead of being hard-coded like the previous. The program also needs to be extensible, allowing ease of adding additional features or different possible medical methods. This could be done either the use of a plug-in system or easy to integrate with code.

The next steps of the project would be making the UI more user friendly. The general ideas around this is possibly separating the forms into different tab groups, adding additional help sections for users that are learning the methods, a simple beginning form thats asks a few basic questions and then chooses the correct form for the user.

\subsection{Project Goals}

The main goals of the project are as follows.

\begin{itemize}

\item Port over the previous code to a modern portable language.
\item Allow the program to read in the measurement data instead of having it hard-coded into the program.
\item Design a simple UI for user interaction (the UI should closely follow the forms that they are used to filling out).

\end{itemize}

Additional goals would be to improve the functionality and the user experience of the program. Some of the discussed examples are as follows.

\begin{itemize}

\item Making a tutorial/help system for newer users still learning the program and/or methods for the measurements.
\item Separate the different forms into their own tab groups.
\item Allow for the different types and forms for bone measurements.
\item Design graphs to show the data in a more understandable format.
\end{itemize}

\end{document}
